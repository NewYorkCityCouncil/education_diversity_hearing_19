\documentclass[11pt,]{article}
\usepackage{lmodern}
\usepackage{amssymb,amsmath}
\usepackage{ifxetex,ifluatex}
\usepackage{fixltx2e} % provides \textsubscript
\ifnum 0\ifxetex 1\fi\ifluatex 1\fi=0 % if pdftex
  \usepackage[T1]{fontenc}
  \usepackage[utf8]{inputenc}
\else % if luatex or xelatex
  \ifxetex
    \usepackage{mathspec}
  \else
    \usepackage{fontspec}
  \fi
  \defaultfontfeatures{Ligatures=TeX,Scale=MatchLowercase}
\fi
% use upquote if available, for straight quotes in verbatim environments
\IfFileExists{upquote.sty}{\usepackage{upquote}}{}
% use microtype if available
\IfFileExists{microtype.sty}{%
\usepackage{microtype}
\UseMicrotypeSet[protrusion]{basicmath} % disable protrusion for tt fonts
}{}
\usepackage{hyperref}
\hypersetup{unicode=true,
            pdftitle={School diversity hearing},
            pdfauthor={Nick Solomon and James Subudhi},
            pdfborder={0 0 0},
            breaklinks=true}
\urlstyle{same}  % don't use monospace font for urls
\usepackage{longtable,booktabs}
\usepackage{graphicx,grffile}
\makeatletter
\def\maxwidth{\ifdim\Gin@nat@width>\linewidth\linewidth\else\Gin@nat@width\fi}
\def\maxheight{\ifdim\Gin@nat@height>\textheight\textheight\else\Gin@nat@height\fi}
\makeatother
% Scale images if necessary, so that they will not overflow the page
% margins by default, and it is still possible to overwrite the defaults
% using explicit options in \includegraphics[width, height, ...]{}
\setkeys{Gin}{width=\maxwidth,height=\maxheight,keepaspectratio}
\IfFileExists{parskip.sty}{%
\usepackage{parskip}
}{% else
\setlength{\parindent}{0pt}
\setlength{\parskip}{6pt plus 2pt minus 1pt}
}
\setlength{\emergencystretch}{3em}  % prevent overfull lines
\providecommand{\tightlist}{%
  \setlength{\itemsep}{0pt}\setlength{\parskip}{0pt}}
\setcounter{secnumdepth}{5}

%%% Use protect on footnotes to avoid problems with footnotes in titles
\let\rmarkdownfootnote\footnote%
\def\footnote{\protect\rmarkdownfootnote}

%%% Change title format to be more compact
\usepackage{titling}

% Create subtitle command for use in maketitle
\newcommand{\subtitle}[1]{
  \posttitle{
    \large#1
    }
}

\setlength{\droptitle}{-2em}

  \title{School diversity hearing}
    \pretitle{\vspace{\droptitle}\huge}
  \posttitle{\par}
    \author{Nick Solomon and James Subudhi}
    \preauthor{\large\emph}
  \postauthor{\par}
      \predate{\large\emph}
  \postdate{\par}
    \date{April 23, 2019}


%%%%%%%%%%%%%%%%%%%%%%%%%%%%%%%%%%%%%%%%%%%%%%%%%%%%%%%%%%%%%%%%%%%%%%%%%%%%%%%
%% NS ADDITIONS %%%%%%%%%%%%%%%%%%%%%%%%%%%%%%%%%%%%%%%%%%%%%%%%%%%%%%%%%%%%%%%
%%%%%%%%%%%%%%%%%%%%%%%%%%%%%%%%%%%%%%%%%%%%%%%%%%%%%%%%%%%%%%%%%%%%%%%%%%%%%%%
\usepackage{fontspec}
\usepackage{graphicx}

\usepackage[headsep=2cm, margin=1in, top=1.25in]{geometry}

\setmainfont{Open Sans}
\newfontfamily\headingfont[]{Georgia}
\usepackage{titlesec}
\titleformat*{\section}{\LARGE\headingfont}
\titleformat*{\subsection}{\Large\headingfont}
\titleformat*{\subsubsection}{\large\headingfont}
\renewcommand{\maketitlehooka}{\headingfont}

\usepackage{fancyhdr}
\pagestyle{fancy}
\rhead{\raisebox{.5\height}{\includegraphics[width=3.5in]{assets/council-logo}}}

%%%%%%%%%%%%%%%%%%%%%%%%%%%%%%%%%%%%%%%%%%%%%%%%%%%%%%%%%%%%%%%%%%%%%%%%%%%%%%%
%% END NS ADDITIONS %%%%%%%%%%%%%%%%%%%%%%%%%%%%%%%%%%%%%%%%%%%%%%%%%%%%%%%%%%%
%%%%%%%%%%%%%%%%%%%%%%%%%%%%%%%%%%%%%%%%%%%%%%%%%%%%%%%%%%%%%%%%%%%%%%%%%%%%%%%
\begin{document}
\maketitle
\thispagestyle{fancy}

\hypertarget{city-overview}{%
\section{City Overview}\label{city-overview}}

This section contains facts about all schools and students in New York City that were included in the 2017 committee report. All data is for the

\hypertarget{race-and-ethnicity}{%
\subsection{Race and Ethnicity}\label{race-and-ethnicity}}

\includegraphics{prelim_report_files/figure-latex/unnamed-chunk-2-1.pdf}
\begin{itemize}
\tightlist
\item
  74.6\% of black and Hispanic students attend a school that is less than 10\% white students.
\item
  34.3\% of white students attend a school with more than 50\% white students
\end{itemize}
\includegraphics{prelim_report_files/figure-latex/unnamed-chunk-4-1.pdf}

\hypertarget{poverty}{%
\subsection{Poverty}\label{poverty}}
\begin{itemize}
\tightlist
\item
  Citywide, 74.0\% of students experience poverty. The citywide Economic Need Index is 70.7\%.
\end{itemize}
\hypertarget{ell-students}{%
\subsection{ELL Students}\label{ell-students}}
\begin{itemize}
\tightlist
\item
  25.2\% of schools have a population of more than 20\% ELL students
\item
  47.4\% of schools have a population of less than 10\% ELL students
\end{itemize}
\hypertarget{students-with-disabilities}{%
\subsection{Students With Disabilities}\label{students-with-disabilities}}
\begin{itemize}
\item
  19.7\% of students in New York City are students with disabilities.
\item
  6.06\% of schools have a population that is less than 10\% students with disabilities.
\end{itemize}
\hypertarget{students-in-temporary-housing}{%
\subsection{Students in temporary housing}\label{students-in-temporary-housing}}
\begin{itemize}
\tightlist
\item
  29.2\% of schools have a population of more than 15\% of students in temporary housing.
\item
  4.63\% of schools do not have any students in temporary housing.
\end{itemize}
\hypertarget{specialized-high-school-diversity}{%
\section{Specialized High School Diversity}\label{specialized-high-school-diversity}}

Specialized high schools have been an important part of the conversation around school diversity and integration in New York City. The graphic below shows the racial demographics of 8 of the 9 specialized high schools. LaGuardia High School is not show here as it has auditions based admissions. Clearly, severe discrepancies exist for those given access to the highest level of academic achievement New York City has to offer.

\includegraphics{prelim_report_files/figure-latex/unnamed-chunk-9-1.pdf}

\hypertarget{poverty-in-specialized-high-schools}{%
\subsection{Poverty in Specialized High Schools}\label{poverty-in-specialized-high-schools}}
\begin{tabular}{l|l|l}
\hline
School & Percent of Students Experiencing Poverty & Economic Need Index\\
\hline
Stuyvesant High School & 44.3\% & 41.8\%\\
\hline
High School for Mathematics, Science and Engineering & 42.6\% & 41.5\%\\
\hline
The Bronx High School of Science & 44.2\% & 39.5\%\\
\hline
High School of American Studies at Lehman College & 20.3\% & 24.1\%\\
\hline
Brooklyn Technical High School & 60.8\% & 52.0\%\\
\hline
The Brooklyn Latin School & 61.7\% & 52.7\%\\
\hline
Queens High School for the Sciences at York College & 60.5\% & 47.1\%\\
\hline
Staten Island Technical High School & 40.8\% & 35.1\%\\
\hline
\end{tabular}
\hypertarget{elementary-school-diversity-and-geography}{%
\section{Elementary School Diversity and Geography}\label{elementary-school-diversity-and-geography}}

Below is a plot showing the racial and ethnic demographics of two co-located elementary schools in District 1. Despite being in literally the same location, these two schools exhibit very different demographics. What decisions made by the school, the district, or the Department of Education have lead to this discrepancy?
\begin{figure}
\centering
\includegraphics{prelim_report_files/figure-latex/unnamed-chunk-12-1.pdf}
\caption{\label{fig:unnamed-chunk-12}Segregated and integrated schools are often very close to one another. These two elementary schools are co-located in the same building, but one is much more segregated than the other.}
\end{figure}
The STAR Academy is predominately black and Hispanic, with fewer than 20\% white students. The Neighborhood school, on the other hand is more than 40\% white. What could cause such a discrepancy? Furthermore, these schools show differences in demographics beyond race. In the 2017-18 school year, The STAR Academy, which is predominantly Hispanic and black, had an economic need index of 75.9\% and the DOE classified 81.8\% of its students as experiencing poverty. In comparison, during the same year The Neighborhood School had an economic need index of 48.0\% and 43.8\% of its students experienced poverty. These schools are divided not just along racial lines but along socio-economic lines.


\end{document}
